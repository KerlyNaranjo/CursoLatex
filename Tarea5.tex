\documentclass[a4paper, 11pt]{article}

% Paquetes
\usepackage[utf8]{inputenc}
\usepackage[spanish]{babel}
\usepackage[top=1cm, left=2cm]{geometry}
\usepackage{color}
\usepackage{multicol}
\usepackage{multirow}

% Comandos
\parindent = 0mm
\author{Kerly Naranjo}
\title{Capítulo 5 Tarea 5}
\date{\today}

% Contenido
\begin{document}
\maketitle
\begin{table}[h]
\caption{Modelos de regresiòn ordinal}
\begin{tabular}{c p{1,6cm} lcccccc} \hline
Modelos & & & B & EE & Wald & p & LI & LS\\ \hline
\multirow{8}{*}{HNI-16} & \multirow{4}{2cm}{Variable predicha}& 1 & 1.34 & 0.54 & 6.24 & .013 & 0.29 \\
& & 3 & 2.61 & 0.56 & 21.80 & \(<\).001 & 1.51 & 3.70 \\
&& 5 & 4.36 & 0.58 & 56.33 & \(<\).001 & 3.22 & 5.50 \\
&& 7 & 7.08 & 0.81 & 77.17 & \(<\).001 & 5.50 & 8.66 \\
& \multirow{4}{2cm}{Variables predictoras} & [Rlg = Católica] & 0.64 & 0.21 & 8.96 & .003 & 0.22 & 1.05 \\
&& [Rlg = Cristiana] & 1.99 & 0.44 & 20.19 & < .001 & 1.12 & 2.86 \\
&& [AmHo = No] & 0.58 & 0.18 & 10.45 & .001 & 0.23 & 0.94 \\
&& [OS = Hetero.] & 1.91 & 0.53 & 13.22 & < .001 & 0.88 & 2.95 \\


\multirow{8}{*}{EXT} & \multirow{4}{2cm}{Variable predicha}& 1 & -0.31 & 0.22 & 1.98 & .159 & -0.75 &0.12 \\
& & 3 & 2.61 & 0.56 & 21.80 & \(<\).001 & 1.51 & 3.70 \\
&& 5 & 4.36 & 0.58 & 56.33 & \(<\) .001 & 3.22 & 5.50 \\
&& 7 & 7.08 & 0.81 & 77.17 & \(<\) .001 & 5.50 & 8.66 \\
& \multirow{4}{2cm}{Variables predictoras} & [Rlg = Católica] & 0.64 & 0.21 & 8.96 & .003 & 0.22 & 1.05 \\
&& [Rlg = Cristiana] & 1.99 & 0.44 & 20.19 & < .001 & 1.12 & 2.86 \\
&& [AmHo = No] & 0.58 & 0.18 & 10.45 & .001 & 0.23 & 0.94 \\
&& [Sexo = Mujer] & 1.91 & 0.53 & 13.22 & < .001 & 0.88 & 2.95 \\


\multirow{8}{*}{INT} & \multirow{4}{2cm}{Variable predicha}& 1 & -0.31 & 0.22 & 1.98 & .159 & -0.75 &0.12 \\
& & 3 & 2.61 & 0.56 & 21.80 & \(<\) .001 & 1.51 & 3.70 \\
&& 5 & 4.36 & 0.58 & 56.33 & \(<\) .001 & 3.22 & 5.50 \\
&& 7 & 7.08 & 0.81 & 77.17 & \(<\) .001 & 5.50 & 8.66 \\

& \multirow{3}{2cm}{Variables predictoras} & [Rlg = Católica] & 0.64 & 0.21 & 8.96 & .003 & 0.22 & 1.05 \\
&& [Rlg = Cristiana] & 1.99 & 0.44 & 20.19 & \(<\) .001 & 1.12 & 2.86 \\
&& [OS = Hetero.] & 1.91 & 0.53 & 13.22 & \(<\) .001 & 0.88 & 2.95 \\

\multirow{8}{*}{PROMI} & \multirow{4}{2cm}{Variable predicha}& 1 & -0.31 & 0.22 & 1.98 & .159 & -0.75 &0.12 \\
& & 3 & 2.61 & 0.56 & 21.80 & \(<\) .001 & 1.51 & 3.70 \\
&& 5 & 4.36 & 0.58 & 56.33 & \(<\) .001 & 3.22 & 5.50 \\
&& 7 & 7.08 & 0.81 & 77.17 & \(<\) .001 & 5.50 & 8.66 \\

& \multirow{4}{2cm}{Variables predictoras} & [Rlg = Católica] & 0.64 & 0.21 & 8.96 & .003 & 0.22 & 1.05 \\
&& [Rlg = Cristiana] & 1.99 & 0.44 & 20.19 & \(<\) .001 & 1.12 & 2.86 \\
&& [OS = Hetero] & 0.58 & 0.18 & 10.45 & .001 & 0.23 & 0.94 \\
&& [VISA = No] & 1.91 & 0.53 & 13.22 & \(<\) .001 & 0.88 & 2.95 \\
\hline
\end{tabular}
\label{tabla:03}
\end{table}
Parámetros fijados a 0: [Religión (Rlg) = Otra], [Amigos homosexuales (AmHo) = Sí], [Orientación
sexual (OS) = No heterosexual], [Sexo = Hombre] e [Inicio de la vida sexual activa de pareja (IVSA)
= Sí]. HNI-16: puntuación total de la escala HNI-16, EXT: rechazo de la manifestación pública de la
homosexualidad, INT: rechazo de los deseos, pensamientos e identidad homosexuales propios,
PROMI: calificación de las personas homosexuales como promiscuas. Valores de las variables
predichas: 1 “completamente en desacuerdo”, 3 “en desacuerdo”, 5 “indiferente”, 7 “de acuerdo” y
9 “definitivamente de acuerdo” (categoría de referencia).
\end{document}