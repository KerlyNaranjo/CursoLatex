\documentclass[a4paper, 12pt]{article}

% Paquetes
\usepackage[utf8]{inputenc}
%\usepackage[T1]{fontenc}
\usepackage[spanish,es-nolayout,es-nodecimaldot,es-tabla]{babel}
\usepackage[left=2.5cm,right=2.5cm,top=2cm,bottom=2cm]{geometry}
\usepackage{amsmath, amssymb, amsfonts, latexsym, amsthm}
\usepackage{listings}

\usepackage{enumerate}
\usepackage{enumitem}
\usepackage{parskip}
\usepackage[colorlinks = true]{hyperref}

\usepackage{graphicx}
\usepackage{color}
\usepackage{multicol}
\usepackage{nicefrac}
\usepackage{upgreek}
\usepackage{cases}

%
\newtheorem{teo}{Teorema}

% Comandos
\newcommand{\R}{\mathbb{R}}
\newcommand{\yds}{\qquad\text{y}\qquad}
\DeclareMathOperator{\proy}{proy}
\DeclareMathOperator{\dd}{d}

%--------------------------------------------------------
%       Datos informativos
%--------------------------------------------------------
\parindent = 0mm
\author{Kerly Naranjo \\ E-mail \href{mailto:kerly.naranjo@epn.edu.ec}{kerly.naranjo@epn.edu.ec}}
\title{MÉTODO DE NEWTON-RAPHSON}
\date{\today}

% Contenido
\begin{document}
	\maketitle


\section*{Introducción}
El método de Newton-Raphson es un método iterativo que nos permite
aproximar la solución de una ecuación del tipo \(f(x)=0\). Partimos de una estimación inicial de la solución \(x0\) y construimos una sucesión de aproximaciones de forma recurrente mediante la fórmula
\[ x_{j+1} = x_j-\dfrac{f(x_j)}{f^1(x_j)} \]
Por ejemplo, consideremos la ecuación
\[e^x=\dfrac{1}{x}\]
En este caso es imposible despejar la incógnita, no obstante, si representamos las curvas \(y = ex\), \(y = 1/x\) en el intervalo \(x \in [0, 4]\), es evidente que la ecuación tiene una solución en este intervalo.

Para aplicar el método de Newton-Raphson, seguimos los siguientes pasos:
\begin{enumerate}
	 \item Expresamos la ecuación en la forma \(f(x)=0\), e identificamos la función \(f\). En el ejemplo es
	 \[f(x)=e^x-\dfrac{1}{x}\]
	 \item Calculamos la derivada
	 \[f'(x)=e^x+\dfrac{1}{x^2}\]
	 \item Construimos la fórmula de recurrencia
	 \[x_{j+1}=x_j-\dfrac{e^x_j-\dfrac{1}{x_j}}{e^x_j+\dfrac{1}{x^{2}_{j}}}\]
	 \item Tomamos una estimación inicial de la solución. En este caso podemos tomar por ejemplo \(x_0 = 1.0\), y calculamos las siguientes aproximaciones. Desde el punto de vista práctico, si deseamos aproximar la solución con 6 decimales, podemos detener los cálculos cuando dos aproximaciones consecutivas coincidan hasta el decimal 8. En nuestro caso, obtendríamos
	 \begin{align}
	 	x_0 &= 1.0, \nonumber \\
	 	x_1 &= 1-\dfrac{e^1-\dfrac{1}{1}}{e^1+\dfrac{1}{1^2}}=0.53788284, \nonumber \\
	 	x_2 &= x_1-\dfrac{e^x_1-\dfrac{1}{x_1}}{e^x_1+\dfrac{1}{x^{2}_{1}}} = 0.56627701, \nonumber \\
	 	x_3 &= 0.56714258, \nonumber \\
	 	x_4 &= 0,56714329, \nonumber \\
	 	x_5 &= 0,56714329. \nonumber
	 \end{align}
	\item Podemos, entonces, tomar como solución \(x = 0.567143\).
\end{enumerate}

\section*{CÓDIGO DE IMPLEMENTACIÓN}
\textbf{Java NetBeans}
\begin{lstlisting}
	/*----------MET. NEWTON RAPHSON---------*/
    public double NewtonRaphson(float x1, double eps, double mxIt){
        float x2;
        for (int k = 1; k <= mxIt; k++) {
            x2 = (float) (x1 - (SP(x1)/DerivadaK(x1, 1)));
            
            if(SP(x2)<=eps){
                return x2;
            }else{
                x1 = x2;
            }
        }
        return 99999999;
    }
\end{lstlisting}

%Referencias
\bibliographystyle{plain}
\bibliography{ref}

\end{document}